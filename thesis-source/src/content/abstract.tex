\pagenumbering{roman}
\setcounter{page}{1}

%----------------------------------------------------------------------------
% Abstract in English
%----------------------------------------------------------------------------
\chapter*{Abstract}\addcontentsline{toc}{chapter}{Abstract}

This paper presents the design and practical implementation of a C code generator module for the Gamma Statechart Composition Framework. The module serves as a vital component in the development of embedded systems software, offering seamless generation of C code from statecharts and compositions of statecharts. By leveraging the existing low-level statechart descriptors, particularly the XSTS language, the code generator module significantly simplifies the process of transforming high-level statechart models into executable C code, simplifying the process of developing embedded systems software. The resulting code is designed to be instantly compilable and can support a diverse range of hardware architectures.

\vfill
\cleardoublepage

%----------------------------------------------------------------------------
% Abstract in Hungarian
%----------------------------------------------------------------------------
\chapter*{Kivonat}\addcontentsline{toc}{chapter}{Abstract}

Ez a dolgozat bemutatja a Gamma Keretrendszer számára fejlesztett C kódgenerátor modulom tervezését és gyakorlati megvalósítását. A modul fontos szerepet játszik a beágyazott rendszerek fejlesztésében, mivel lehetővé teszi C kód generálását állapotgépekből és ezek kompozícióiból. A már meglévő alacsony szintű állapotgép-leírókat, különösen az XSTS nyelvet kihasználva a kódgenerátor modul jelentősen leegyszerűsíti a magas szintű modellek transzformációját fordítható C kóddá. Az elkészült kód gond nélkül fordítható a támogatott platformokon.

\vfill
\cleardoublepage

\selectthesislanguage

\newcounter{romanPage}
\setcounter{romanPage}{\value{page}}
\stepcounter{romanPage}