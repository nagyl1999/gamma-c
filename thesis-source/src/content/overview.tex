% !TeX spellcheck = en_EN
% !TeX encoding = UTF-8
% !TeX program = xelatex
%----------------------------------------------------------------------------
\chapter{Modeling and Code Generation}
%----------------------------------------------------------------------------

This chapter covers various aspects related to modeling and code generation. It begins by discussing the difficulties that arise when generating code from low-level model descriptors for embedded systems. The chapter then goes on to cover the Gamma statecharts and its composition language, model transformations, and mapping XSTS to C code.

%----------------------------------------------------------------------------
\section{Code Generation for Embedded Systems}
%----------------------------------------------------------------------------

This section discusses the challenges that arise when generating code from low-level models. Embedded systems software is required to run on various hardware architectures to ensure flexibility and portability on different hardware components, reduce constraints on physical device selection, and provide long term support. Therefore, a high-level design language is needed to model the reactive system's behavior independently of the hardware and to enable software development on multiple platforms. 

UML Statecharts serve best as the high-level design language for modeling component based reactive systems due to their ability to capture complex state behavior and event handling, greatly reducing complexity compared to programmer-made code. Gamma provides a composite language to build compositions of statecharts which helps to reduce complexity and increase modularity. As of now, Gamma lacks the capability to generate platform independent code apart from the already existing Java code generator which is not optimal for embedded systems due to its high resource requirements and lack of real-time support. The following objectives should be met by a code generator intended for embedded systems:

\begin{itemize}
	\item \textbf{Limited resources:} The code should be runnable on wide-range of hardware architectures. Its choice of language should consider resource requirements and compiler support for embedded hardware.
	
	\item \textbf{Portability:} The generated code should be portable across different platforms to allow for flexibility and use among different domains, from the use of embedded Linux to programming microcontrollers.
	
	\item \textbf{Safety and reliability:} Embedded systems are often used in critical applications such as medical devices and transportation systems, therefore the generated code should be safe and reliable.
\end{itemize}

Considering the aforementioned requirements, I have opted for the C programming language as it demands minimal resources and has extensive support from compilers used for embedded hardware along with software verification tools.

%----------------------------------------------------------------------------
\section{Gamma Statecharts}
%----------------------------------------------------------------------------

In Gamma, components are defined using statecharts, which are based on the Unified Modeling Language (UML). Statecharts consist of states that define the state of the component as well as how it reacts to incoming events. Events are responsible for initiating a possible state change and might have parameters linked to it. Events can come from outside the component, or can be produced by internal timers. Actions are executable statements that are executed when a transition occurs, while guards are conditions that must be met for a transition to occur. Timeout events can be associated with states and transitions in statecharts, which trigger after a specified duration of time. Upon entering or exiting a state, entry and exit actions can be executed. These actions have the ability to fire events. UML Statecharts also include hierarchical states, which allow for the composition of multiple states into a higher-level state, and the ability to nest states within other states. Shallow and deep history states store the last active state of a higher-level state when it is exited. When re-entering the high-level state, the active inner state will be the one stored by the history state in case of deep history.

Statecharts provide a high-level representation of a system's logic. While manual implementation of this behavior is possible, it is error-prone and time-consuming. Automatic code generation can eliminate these issues, but it requires a bridge between the high-level statechart model and the low-level programming language. Gamma addresses this issue by transforming its statechart components into low-level representations, which enables developers to work with high level models, such as statecharts, while code generators can utilize the generated low-level representations.

%----------------------------------------------------------------------------
\section{Model Transformation}
%----------------------------------------------------------------------------


\newpage

%----------------------------------------------------------------------------
\section{Mapping XSTS to C Code}
%----------------------------------------------------------------------------

Once a statechart model has been transformed to its low-level descriptor, it needs to be translated to C language to be compiled and executed on the target platform. In the context of this paper, the XSTS model is transformed into C code. This process requires mapping the elements and behavior of the XSTS language onto constructs and syntax within the C language. The table presented below illustrates the mapping between the elements of the XSTS language and the C programming language: \newline

\begin{table}[h]
	\centering
	\begin{tabular}{p{0.20\linewidth\centering}|p{0.20\linewidth}|p{0.40\linewidth}}
		\textbf{XSTS Element} & \textbf{C Language} & \textbf{Comment} \\
		\hline
		type & enum & description of states \\
		integer & int & - \\
		boolean & bool & stdbool library \\
		\hline
		var & declaration & variable declarations \\
		ctrl & - & no corresponding C element \\
		local & - & local variable \\
		assignment & assignment & - \\
		havoc & - & nondeterministic value \\
		assume & if, switch & guard conditions \\
		\hline
		choice & routine & nondeterministic choice \\
		parallel & routine & parallel composition of statements \\
		sequence & routine & sequence of statements \\
		\hline
		env & - & transitions to environment \\
		init & - & initial transitions \\
		trans & - & internal transitions \\
		\hline
	\end{tabular}
	\caption{XSTS Elements mapped to C Language.}
	\label{table-label}
\end{table}

There are a few elements that are not included in the table above because they correspond without the need for special transformation from one language to another. These keywords are the following: \textit{while, for, if, else, and, or}.

